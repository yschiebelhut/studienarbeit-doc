\chapter{Fazit}
Ziel dieser Arbeit war es, ein vorliegendes Reinforcement Learning Problem, bei dem einem spinnenartiger Roboter das Laufen beigebracht wird, um Pfadplanung und das Umsteuern von Hindernissen zu erweitern.
Dazu wurden zunächst wichtige Grundlagen erläutert und der State of the Art von Reinforcement Learning gesteuerter Lokomotion vorgestellt.
Anschließend wurde ein Konzept zur Lösung der Aufgabe erarbeitet.
Dabei wurden vor allem notwendige Modifikationen am Roboter und Möglichkeiten zur Pfadplanung diskutiert.
Die gewählte Realisierung soll dem Roboter generische Positionierungsinformationen bereitstellen.
Weiterhin wird ein Pfad nicht als ganzes betrachtet, sondern in einzelne Punkte zerlegt.
Das Erreichen des jeweils nächsten Punktes kann im Kern auf das Ausgangsproblem reduziert werden.

Gemäß dem entwickelten Konzept wurde die Trainingsumgebung an den aktuellen Stand der Technik angepasst und verschiedene Testreihen mit dem \acl{ppo}-Algorithmus durchgeführt.
Dabei wurde erfolgreich das zuvor sprungbasierte Laufverhalten des Roboters stabilisiert.
Für die Pfadplanung wurde, anhand eines ähnlich aufgebauten Projekts, erfolgreich ein Proof of Concept implementiert.
Das Training am eigenen Roboter war nur eingeschränkt erfolgreich, wobei als primäre Fehlerquellen die Hyperparameter des Trainings und ein Mangel an zur Verfügung gestellten Informationen vermutet werden.
Die Trainingsdurchläufe wurden anhand geeigneter Metriken analysiert und Vorschläge für zukünftige Verbesserungen dargelegt.
Ein sinnvolles Training der Hindernisumfahrung konnte nicht durchgeführt werden, da dieses auf der Verfolgung eines vorgegebenen Pfads aufbaut.
Trotz dieser Schwierigkeiten wurden mehrere Tests durchgeführt, um die Möglichkeiten der Hindernisumfahrung zu evaluieren und die weitere Roadmap zu beschreiben, wenn eine funktionierende Pfadplanung als Basis erreicht wurde.