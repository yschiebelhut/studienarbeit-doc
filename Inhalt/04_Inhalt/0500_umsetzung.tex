\chapter{Umsetzung}
\section{Rekonstruktion und Migration der Simulationsumgebung}
Der Quellcode der alten Arbeit liegt auf GitHub\footnote{\url{https://github.com/MobMonRob/HindernisumfahrungRLStudien/tree/1c0a884c2133107562e4928bfa8bef2ee6e2ade0}} vor.
Im ersten Schritt wird angestrebt, diesen Arbeitsstand zu rekonstruieren, sodass ein Training in der Unity-Umgebung möglich ist.

\subsection{Programmversionen}
Zur Projektbasis liegt neben der Ausarbeitung in \cite{waidner.2020} nur Quellcode vor.
Leider werden dabei die verwendeten Programmversionen nicht dokumentiert, welche jedoch notwendigerweise fein aufeinander abgestimmt sein müssen, um ein Training zu ermöglichen und dessen Erfolg zu gewährleisten.
Sowohl Unity als auch ML-Agents und das zugehörige Python-Toolkit wurden seit der Durchführung von \cite{waidner.2020} in unterschiedlicher Geschwindigkeit weiterentwickelt.
Aufgrund von auftretenden Inkompatibilitäten empfiehlt es sich daher, zunächst die Originalversionen aufzusetzen und dies als Ausgangspunkt für weitere Anpassungen zu nutzen.
Die Unity-Projektdatei enthält Informationen über die exakte Unity-Version, die zur Erstellung des Projekts verwendet wurde.
Leider fehlt jedoch Dokumentation zur verwendeten Version von ML-Agents und des Python-Toolkits.
Recherche in der Veröffentlichungshistorie von ML-Agents ergeben, das höchstwahrscheinlich Version 0.14.1 des Unity-Plugins verwendet wurde.
Daraus ergeben sich Python-Abhängigkeiten, die darauf hindeuten, dass eine Python Version \textgreater= 3.6 und \textless 3.7 verwendet wurde.
Leider gestaltet sich der Versuch erfolglos, diese Versionen auf den verwendeten Entwicklungssystemen (OS X und Linux) zu installieren.
Somit ist es auch nicht möglich, die exakte Entwicklungskonstellation von \cite{waidner.2020} zu rekonstruieren.

Um die Ergebnisse dieser Arbeit möglichst nachhaltig zu machen, soll zur weiteren Entwicklung die aktuelle Version von ML-Agents verwendet werden (Unity-Paket in der Version 2.0.1, Stand: Mai 2023).
Das korrespondierende Python-Plugin ist \code{mlagents} in der Version 0.30.0.
(Zum Zeitpunkt des Schreibens ist es notwendig, das Paket \code{protobuf} in der Version 3.20.3 explizit zu installieren, da sonst die Installation von \code{mlagents} scheitert.)
Die Dependencies dieses Pakets bedingen, dass als neuste Python-Version 3.10.8 verwendet werden kann, welche auch zur Entwicklung gewählt wird.

(Installation und Management spezifischer Patch-Versionen von Python kann kompliziert sein, da in der Regel ein Versionsmanagement nur anhand der Minor-Versionen vorgesehen ist.
Im Rahmen dieser Arbeit hat sich der Einsatz des Programms \code{pyenv}\footnote{\url{https://github.com/pyenv/pyenv}} empfohlen, da sich damit sehr einfach spezifische Versionen von Python individuell kompilieren und managen lassen.)

Da es einerseits Problematiken bereiten kann, neue Plugins mit einer alten Version von Unity zu betreiben und zusätzlich die damals verwendete Version massive Fehler in Kombination mit OS X aufweist, wird auch Unity auf eine aktuelle Version angepasst.
Dafür wird in diesem Fall die aktuellste LTS-Version zum Zeitpunkt der Entwicklung verwendet (2021.3.21.f1).

\subsection{Änderungen der Codebasis}
Da es sich bei ML-Agents um ein vergleichsweise neues Toolkit handelt, unterliegt es fortlaufend einer starken Entwicklung.
Im Zeitraum seit der Vorgängerarbeit wurde das Plugin von einer Alphaversion zu einem offiziellen Release gebracht.
In diesen Entwicklungsphasen kommt es bei Software häufig zu Breaking Changes.
Auch bei ML-Agents ist dies der Fall und es kommt umgehend zum Kompilierungsfehlern, wenn das Unity-Projekt im Editor geöffnet wird.
Der erste Schritt besteht deshalb darin, herauszufinden, welche Methoden davon betroffen sind.
Dafür werden im Fehlerbericht die Methoden gesucht, die Fehler enthalten.
Die Methodenköpfe können dann in die Versionshistorie von ML-Agents\footnote{\url{https://github.com/Unity-Technologies/ml-agents/releases}} gesucht werden.
Im vorliegenden Fall sind somit alle an der Schnittstelle des Toolkit vorgenommenen Änderungen deutlich aufgeschlüsselt und geben Aufschluss darüber, wie die Kompatibilität des Quellcodes wiederhergestellt werden kann.

Für das Training wird in \cite{waidner.2020} eine Trainer-Config-YAML-Datei verwendet, wie sie in \autoref{sec:training} beschrieben wird.
Mit einer Aktualisierung des Python-Toolkits hat sich auch das Format dieser Datei verändert, weshalb die ursprüngliche Datei nicht mehr kompatibel zum nun verwendeten Tooling ist.
Die Änderungen am Format sind vergleichsweise gering und die Datei von überschaubarer Größe, weshalb nach dem Vorbild der alten Datei und unter Anleitung von \cite{mlagentsHyperparameter} die Datei neu gebaut werden kann.
Der Codestand ist nach diesen Veränderungen\footnote{\url{https://github.com/MobMonRob/HindernisumfahrungRLStudien/tree/ec8abbd161217c9a42adb42779e01c5b3dfeb209}} nun mit den aktualisierten Versionen des Toolings kompatibel und wird als Ausgangspunkt für die Experimente dieser Arbeit verwendet.

\subsection{Durchführung des Trainings}
Um ein Training in der Simulationsumgebung durchzuführen wird zuerst eine kompilierte Binary der Trainingsumgebung erstellt.
Zwar ist es theoretisch möglich, direkt aus dem Unity-Editor das Training zu starten.
Allerdings bringt dies einige Nachteile mit sich, wie etwa mangelnde Skalierbarkeit und Performanceverluste.
Außerdem ist es so nicht möglich, das Training effizient auf einer unabhängigen Maschine durchzuführen, die eine weit höhere Trainingsrate ermöglicht.
Um eine solche Binary zu erstellen, wird der Build Prozess in Unity unter File \textgreater Build Settings \textgreater Build gestartet.
Dabei kann die gewünschte Zielplattform ausgewählt werden -- in Abhängigkeit, wo das Training durchgeführt werden soll.
Zum Cross-Compiling ist es allerdings vorausgesetzt, dass die entsprechenden Erweiterungen und Bibliotheken bei der Installation von Unity ausgewählt und geladen wurden.
Sonst ist standardmäßig nur das Kompilieren für die Architektur und Plattform möglich, unter der der Editor ausgeführt wird.