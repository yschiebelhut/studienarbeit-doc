\renewcommand{\abstractname}{Abstract} % Veränderter Name für das Abstract
\begin{abstract}
\begin{addmargin}[1.5cm]{1.5cm}        % Erhöhte Ränder, für Abstract Look
\thispagestyle{plain}                  % Seitenzahl auf der Abstract Seite

\begin{center}
\small\textit{- English -}             % Angabe der Sprache für das Abstract
\end{center}

\vspace{0.25cm}

% LTeX: language=en-US

The manual implementation of a control system for multi-legged robots is highly complex.
For this reason, a previous study investigated how a spider-like robot can learn to walk using reinforcement learning without requiring advanced anatomical knowledge or a great deal of effort.
The trained model only allows movement in one direction.

\vspace{0.25cm}

This thesis explores the extension of this model to allow the ability to avoid obstacles along a given path.
The training is carried out in an environment simulated with Unity, both to speed it up and because transferring it to reality would require modifications to the robot.
For the training, the \acl{ppo} algorithm is used, which has shown the best results of all compared algorithms in the previous work.

\vspace{0.25cm}

The walking behavior of the robot is successfully stabilized.
A proof of concept is developed for path planning, which demonstrates the suitability of the developed approach.
Based on this concept, various test series for path planning and obstacle avoidance are carried out and their results are analyzed.

\end{addmargin}
\end{abstract}