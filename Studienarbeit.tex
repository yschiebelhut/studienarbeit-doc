% ------------------------------------------------------------
% LaTeX Template für die DHBW zum Schnellstart!
% Original: https://github.wdf.sap.corp/vtgermany/LaTeX-Template-DHBW
% ------------------------------------------------------------
% ---- Präambel mit Angaben zum Dokument
% LTeX: enabled=false

\documentclass[
	fontsize=12pt,           % Leitlinien sprechen von Schriftgröße 12.
	paper=A4,
	twoside=false,
	listof=totoc,            % Tabellen- und Abbildungsverzeichnis ins Inhaltsverzeichnis
	bibliography=totoc,      % Literaturverzeichnis ins Inhaltsverzeichnis aufnehmen
	titlepage,               % Titlepage-Umgebung anstatt \maketitle
	headsepline,             % horizontale Linie unter Kolumnentitel
	abstract,              % Überschrift einschalten, Abstract muss in {abstract}-Umgebung stehen
]{scrreprt}                  % Verwendung von KOMA-Report
\usepackage[utf8]{inputenc}  % UTF8 Encoding einschalten
\usepackage[ngerman]{babel}  % Neue deutsche Rechtschreibung
% \usepackage[T1]{fontenc}     % Ausgabe von westeuropäischen Zeichen (auch Umlaute)
% \usepackage{fontspec}
\usepackage{microtype}       % Trennung von Wörtern wird besser umgesetzt
\usepackage{lmodern}         % Nicht-gerasterte Schriftarten (bei MikTeX erforderlich)
\usepackage{graphicx}        % Einbinden von Grafiken erlauben
\usepackage{rotating}        % Rotieren von Grafiken ermöglichen
\usepackage{adjustbox}
\usepackage{wrapfig}         % Grafiken fließend im Text
\usepackage{setspace}        % Zeilenabstand \singlespacing, \onehalfspaceing, \doublespacing
\usepackage[
	%showframe,                % Ränder anzeigen lassen
	left=2.7cm, right=2.5cm,
	top=2.5cm,  bottom=2.5cm,
	includeheadfoot,
	a4paper
]{geometry}                      % Seitenlayout einstellen
\usepackage{scrlayer-scrpage}    % Gestaltung von Fuß- und Kopfzeilen
\usepackage[printonlyused]{acronym}             % Abkürzungen, Abkürzungsverzeichnis
\usepackage{titletoc}            % Anpassungen am Inhaltsverzeichnis
\contentsmargin{0.75cm}          % Abstand im Inhaltsverzeichnis zw. Punkt und Seitenzahl
\usepackage[                     % Klickbare Links (enth. auch "nameref", "url" Package)
  hidelinks,                     % Blende die "URL Boxen" aus.
  breaklinks=true                % Breche zu lange URLs am Zeilenende um
]{hyperref}
\usepackage[hypcap=true]{caption}% Anker Anpassung für Referenzen
\urlstyle{same}                  % Aktuelle Schrift auch für URLs
% Anpassung von autoref für Gleichungen (ergänzt runde Klammern) und Algorithm.
% Anstatt "Listing" kann auch z.B. "Code-Ausschnitt" verwendet werden. Dies sollte
% jedoch synchron gehalten werden mit \lstlistingname (siehe weiter unten).
\addto\extrasngerman{%
	\def\equationautorefname~#1\null{Gleichung~(#1)\null}
	\def\lstnumberautorefname{Zeile}
	\def\lstlistingautorefname{Listing}
	\def\algorithmautorefname{Algorithmus}
	% Damit einheitlich "Abschnitt 1.2[.3]" verwendet wird und nicht "Unterabschnitt 1.2.3"
	% \def\subsectionautorefname{Abschnitt}
}

% ---- Abstand verkleinern von der Überschrift 
\renewcommand*{\chapterheadstartvskip}{\vspace*{.5\baselineskip}}

% Hierdurch werden Schusterjungen und Hurenkinder vermieden, d.h. einzelne Wörter
% auf der nächsten Seite oder in einer einzigen Zeile.
% LaTeX kann diese dennoch erzeugen, falls das Layout ansonsten nicht umsetzbar ist.
% Diese Werte sind aber gute Startwerte.
\widowpenalty10000
\clubpenalty10000

% ---- Für das Quellenverzeichnis
\usepackage[
	backend = biber,                % Verweis auf biber
	language = auto,
	style = numeric,                % Nummerierung der Quellen mit Zahlen
	sorting = none,                 % none = Sortierung nach der Erscheinung im Dokument
	sortcites = true,               % Sortiert die Quellen innerhalb eines cite-Befehls
	block = space,                  % Extra Leerzeichen zwischen Blocks
	hyperref = true,                % Links sind klickbar auch in der Quelle
	%backref = true,                % Referenz, auf den Text an die zitierte Stelle
	bibencoding = auto,
	giveninits = true,              % Vornamen werden abgekürzt
	doi=false,                      % DOI nicht anzeigen
	isbn=false,                     % ISBN nicht anzeigen
    alldates=short                  % Datum immer als DD.MM.YYYY anzeigen
]{biblatex}
\addbibresource{library.bib}
\setcounter{biburlnumpenalty}{3000}     % Umbruchgrenze für Zahlen
\setcounter{biburlucpenalty}{6000}      % Umbruchgrenze für Großbuchstaben
\setcounter{biburllcpenalty}{9000}      % Umbruchgrenze für Kleinbuchstaben
\DeclareNameAlias{default}{family-given}  % Nachname vor dem Vornamen
\AtBeginBibliography{\renewcommand{\multinamedelim}{\addslash\space
}\renewcommand{\finalnamedelim}{\multinamedelim}}  % Schrägstrich zwischen den Autorennamen
\DefineBibliographyStrings{german}{
  urlseen = {Einsichtnahme:},                      % Ändern des Titels von "besucht am"
}
\usepackage[babel,german=quotes]{csquotes}         % Deutsche Anführungszeichen + Zitate


% ---- Für Mathevorlage
\usepackage{amsmath}    % Erweiterung vom Mathe-Satz
\usepackage{amssymb}    % Lädt amsfonts und weitere Symbole
\usepackage{MnSymbol}   % Für Symbole, die in amssymb nicht enthalten sind.

% ---- Für chemische Formeln
\usepackage{mhchem}

% ---- Für Quellcodevorlage
\usepackage{scrhack}                    % Hack zur Verw. von listings in KOMA-Script
\usepackage{listings}                   % Darstellung von Quellcode
\usepackage{xcolor}                     % Einfache Verwendung von Farben
\input{Inhalt/00_Latex/quellcodeStyle}  % Weitere Details sind ausgelagert

\usepackage{algorithm}                  % Für Algorithmen-Umgebung (ähnlich wie lstlistings Umgebung)
\usepackage{algpseudocode}              % Für Pseudocode. Füge "[noend]" hinzu, wenn du kein "endif",
                                        % etc. haben willst.

\makeatletter                           % Sorgt dafür, dass man @ in Namen verwenden kann.
                                        % Ansonsten gibt es in der nächsten Zeile einen Compilefehler.
\renewcommand{\ALG@name}{Algorithmus}   % Umbenennen von "Algorithm" im Header der Listings.
\makeatother                            % Zeichen wieder zurücksetzen
\renewcommand{\lstlistingname}{Listing} % Erlaubt das Umbenennen von "Listing" in anderen Titel.

% ---- Tabellen
\usepackage{booktabs}  % Für schönere Tabellen. Enthält neue Befehle wie \midrule
\usepackage{multirow}  % Mehrzeilige Tabellen
\usepackage{siunitx}   % Für SI Einheiten und das Ausrichten Nachkommastellen
\sisetup{locale=DE, range-phrase={~bis~}, output-decimal-marker={,}} % Damit ein Komma und kein Punkt verwendet wird.
\usepackage{xfrac} % Für siunitx Option "fraction-function=\sfrac"

% ---- Für Definitionsboxen in der Einleitung
\usepackage{amsthm}                     % Liefert die Grundlagen für Theoreme
\usepackage[framemethod=tikz]{mdframed} % Boxen für die Umrandung
\input{Inhalt/00_Latex/highlightBoxen}  % Weitere Details sind ausgelagert

% ---- Für Todo Notes
\usepackage{todonotes}
\setlength {\marginparwidth }{2cm}      % Abstand für Todo Notizen

\lstdefinelanguage{JavaScript}{
  morekeywords=[1]{break, case, catch, continue, debugger, default, delete, do, else, finally, for, function, if, in, instanceof, new, return, switch, this, throw, try, typeof, var, void, while, with, require, then, const},
  morecomment=[l]{//},
  morecomment=[s]{/*}{*/},
  morestring=[b]{'},
  morestring=[b]{"},
  morestring=[b]{`},
%   alsoletter={\\'},
  stringstyle=\color[rgb]{0.3,0.5,0.4},
  sensitive=true
}

\lstdefinelanguage{Docker}{
	% Keywords as defined in the language grammar
	morekeywords=[1]{%
	  FROM,RUN,COPY,WORKDIR,LABEL,USER,ENTRYPOINT,EXPOSE,CMD},
	% Built-in functions
	morekeywords=[2]{},
	% Pre-declared types
	morekeywords=[3]{},
	% Constants and zero value
	morekeywords=[4]{},
	% Strings : "foo", 'bar', `baz`
	morestring=[b]{"},
	morestring=[b]{'},
	morestring=[b]{`},
	% Comments : /* cpmment */ and // comment
	comment=[l]{\#},
	morecomment=[s]{/*}{*/},
	stringstyle=\color[rgb]{0.3,0.5,0.4},
	% Options
	sensitive=true
}

\usepackage{xurl}
\usepackage{tikz}
\usepackage{svg}

% ---- Elektronische Version oder Gedruckte Version?
% ---- Unterschied: Die elektronische Version enthält keinen Platzhalter für die Unterschrift
\usepackage{ifthen}
\newboolean{e-Abgabe}
\setboolean{e-Abgabe}{false}    % false=gedruckte Fassung

% ---- Persönlichen Daten:
\newcommand{\titel}{Erlernen von Hindernisumfahrung mithilfe von Reinforcement Learning}
\newcommand{\titelheader}{RL-Hindernisumfahrung}
\newcommand{\arbeit}{Studienarbeit (T3\_3101)}
\newcommand{\studiengang}{Informatik}
\newcommand{\studienjahr}{2020}
\newcommand{\autor}{Yannik Schiebelhut}
\newcommand{\autorReverse}{Schiebelhut, Yannik}
\newcommand{\verfassungsort}{Karlsruhe}
\newcommand{\matrikelnr}{3354235}
\newcommand{\kurs}{TINF20B1}
\newcommand{\bearbeitungsmonat}{Mai 2023}
\newcommand{\abgabe}{22. Mai 2023}
\newcommand{\bearbeitungszeitraum}{14.10.2022 - 22.05.2023}
\newcommand{\firmaName}{SAP SE}
\newcommand{\firmaStrasse}{Dietmar-Hopp-Allee 16}
\newcommand{\firmaPlz}{69190 Walldorf, Deutschland}
\newcommand{\betreuerFirma}{Alexander Schaefer}
\newcommand{\betreuerDhbw}{Florian Stöckl}

\input{Inhalt/00_Latex/kopfundFusszeile}

% ---- Hilfreiches
\newcommand{\zB}{z.\,B. }   % "z.B." mit kleinem Leeraum dazwischen (ohne wäre nicht korrekt)
\newcommand{\dash}{d.\,h. }
\newcommand{\Dash}{D.\,h. }

\newcommand{\code}[1]{\texttt{#1}} % Ist einfacher zu schreiben als ständig \texttt und erlaubt
                                   % Änderungen im Nachhinein, wenn man z.B. Inline-Code anders stylen möchte.

% ---- Silbentrennung (falls LaTeX defaults falsch / nicht gewünscht sind)
\hyphenation{HANA}         % anstatt HA-NA
\hyphenation{Graph-Script} % anstatt GraphS-cript

% ---- Beginn des Dokuments
\begin{document}
\setlength{\parindent}{0pt}              % Keine Paragraphen Einrückung.
                                         % Dafür haben wir den Abstand zwischen den Paragraphen.
\setcounter{secnumdepth}{2}              % Nummerierungstiefe fürs Inhaltsverzeichnis
\setcounter{tocdepth}{1}                 % Tiefe des Inhaltsverzeichnisses. Ggf. so anpassen,
                                         % dass das Verzeichnis auf eine Seite passt.
\sffamily                                % Serifenlose Schrift verwenden.

% ---- Vorspann
% ------ Titelseite
\singlespacing
\thispagestyle{empty}
\begin{titlepage}
\enlargethispage{4cm}

\begin{figure}
	\includegraphics[height=2.5cm]{Bilder/Logos/Logo_DHBW.pdf}
	\centering
\end{figure} 
\vspace*{0.1cm}

\begin{center}
	\huge{\textbf{\titel}}\\[1.5cm]
	\Large{\textbf{\arbeit}}\\[0.5cm]
	\normalsize{im Rahmen der Prüfung zum\\[1ex] \textbf{Bachelor of Science (B.Sc.)}}\\[0.5cm]
	\Large{des Studienganges \studiengang}\\[1ex]
	\normalsize{an der Dualen Hochschule Baden-Württemberg Karlsruhe}\\[1cm]
	\normalsize{von}\\[1ex] \Large{\textbf{\autor}} \\[1cm]
\end{center}

\begin{center}
	\vfill
	\begin{tabular}{ll}
		Abgabedatum:                     & \abgabe \\[0.2cm]
		Bearbeitungszeitraum:            & \bearbeitungszeitraum \\[0.2cm]
		Matrikelnummer, Kurs:            & \matrikelnr , \kurs \\[0.2cm]
		% Ausbildungsfirma:                & \firmaName \\
		%                                  & \firmaStrasse \\
		%                                  & \firmaPlz \\[0.2cm]
		% Betreuer der Ausbildungsfirma:   & \betreuerFirma \\[0.2cm]
		Gutachter der Dualen Hochschule: & \betreuerDhbw \\[2cm]
	\end{tabular} 
\end{center}
\end{titlepage}
  % Titelseite
\newcounter{savepage}
\pagenumbering{Roman}                    % Römische Seitenzahlen
\onehalfspacing

% ------ Erklärung, Sperrvermerk, Abstact
\include{Inhalt/01_Standard/erklaerung}
\renewcommand{\abstractname}{Abstract} % Veränderter Name für das Abstract
\begin{abstract}
\begin{addmargin}[1.5cm]{1.5cm}        % Erhöhte Ränder, für Abstract Look
\thispagestyle{plain}                  % Seitenzahl auf der Abstract Seite

\begin{center}
\small\textit{- Deutsch -}             % Angabe der Sprache für das Abstract
\end{center}

\vspace{0.25cm}

Die manuelle Implementierung einer Steuerung für mehrbeinige Roboter ist äußerst komplex.
Aus diesem Grund wurde in einer früheren Studienarbeit untersucht, wie ein spinnenartiger Roboter mittels Reinforcement Learning das Laufen lernen kann, ohne dass fortgeschrittene anatomische Kenntnisse oder ein großer Aufwand erforderlich sind.
Das dabei trainierte Modell ermöglicht nur die Bewegung in eine Richtung.

\vspace{0.25cm}

Im Rahmen dieser Arbeit wird die Erweiterung dieses Modells erforscht, um die Fähigkeit des Ausweichens vor Hindernissen entlang eines vorgegebenen Pfades zu ermöglichen.
Das Training wird in einer mit Unity simulierten Umgebung durchgeführt, um es einerseits zu beschleunigen und andererseits, da eine Übertragung in die Realität Modifikationen am Roboter erfordern würde.
Für das Training wird der \acl{ppo}-Algorithmus verwendet, der in der vorangegangenen Arbeit die besten Ergebnisse aller verglichenen Algorithmen gezeigt hat.

\vspace{0.25cm}

Das Laufverhalten des Roboters wird erfolgreich stabilisiert.
Für die Pfadplanung wird ein Proof of Concept entwickelt, welches die Eignung des entwickelten Ansatzes demonstriert.
Basierend auf diesem Konzept werden verschiedene Testreihen zur Pfadplanung und Hindernisumfahrung durchgeführt und deren Ergebnisse analysiert.


\end{addmargin}
\end{abstract}
\renewcommand{\abstractname}{Abstract} % Veränderter Name für das Abstract
\begin{abstract}
\begin{addmargin}[1.5cm]{1.5cm}        % Erhöhte Ränder, für Abstract Look
\thispagestyle{plain}                  % Seitenzahl auf der Abstract Seite

\begin{center}
\small\textit{- English -}             % Angabe der Sprache für das Abstract
\end{center}

\vspace{0.25cm}

% LTeX: language=en-US

Placeholder

\end{addmargin}
\end{abstract}

% ------ Inhaltsverzeichnis
\singlespacing
\tableofcontents

% ------ Verzeichnisse
\renewcommand*{\chapterpagestyle}{plain}
\pagestyle{plain}
% \include{Inhalt/03_Verzeichnisse/formelgroessen}
\chapter*{Abkürzungsverzeichnis}
\addcontentsline{toc}{chapter}{Abkürzungsverzeichnis} % Hinzufügen zum Inhaltsverzeichnis 

\begin{acronym}[PPOOO] % längstes Kürzel wird verw. für den Abstand zw. Kürzel u. Text

	% Alphabetisch selbst sortieren - nicht verwendete Kürzel rausnehmen!
	% \acro{AIR}{Adobe Integrated Runtime}
	% \acro{AJAX}{Asynchronous Javascript and XML}
	% \acro{ANSI}{American National Standards Institute}
	% \acro{api}[API]{Application Programming Interface}
	% \acro{AR}{Augmented Reality}
	% \acro{BAPI}{Business Application Programming Interface}
	% \acro{BIOS}{Basic Input Output System}
	% \acro{bpmn}[BPMN]{Business Process Model and Notation}
	% \acro{ccwf}[ccWF]{Cross-Company Workflow}
	% \acro{CDMA}{Code Division Multiple Access}
	\acro{cli}[CLI]{Command Line Interface}
	% \acro{csv}[CSV]{Comma-Separated Values}
	% \acro{erp}[ERP]{Enterprise Resource Planning}
	% \acro{hs2}[HS2]{High Speed 2}
	% \acro{HTTP}{Hypertext Transfer Protocol}
	% \acro{HTTPS}{Hypertext Transfer Protocol Secure}
	% \acro{IP}{Internet Protocol}
	% \acro{ISBN}{Internationale Standardbuchnummer}
	% \acrodefplural{ISBN}[ISBNs]{Internationale Standardbuchnummern}
	% \acro{JSON}{JavaScript Object Notation}
	% \acro{kpi}[KPI]{Key Performance Indicator}
	\acro{mdp}[MDP]{Markov Decision Process}
	% \acro{OData}{Open Data Protocol}
	\acro{ppo}[PPO]{Proximal Policy Optimization}
	\acro{sac}[SAC]{Soft Actor-Critic}
	% \acro{SDK}{Software Development Kit}
	% \acro{SEO}{Search Engine Optimization}
	% \acro{sql}[SQL]{Structured Query Language}
	% \acro{SSH}{Secure Shell}
	% \acro{UEFI}{Unified Extensible Firmware Interface}
	% \acro{URI}{Uniform Resource Identifier}
	% \acro{USB}{Universal Serial Bus}
	% \acro{VLAN}{Virtual Local Area Network}
	% \acro{WYSISWG}{What You See Is What You Get}
	% \acro{xes}[XES]{eXtensible Event Stream}
	% \acro{XSL}{Extensible Stylesheet Language}

\end{acronym}
\listoffigures                          % Erzeugen des Abbildungsverzeichnisses 
% \listoftables                           % Erzeugen des Tabellenverzeichnisses
\renewcommand{\lstlistlistingname}{Quellcodeverzeichnis}
\lstlistoflistings                      % Erzeugen des Listenverzeichnisses
\setcounter{savepage}{\value{page}}


% ---- Inhalt der Arbeit
\cleardoublepage
\pagenumbering{arabic}                  % Arabische Seitenzahlen für den Hauptteil
\setlength{\parskip}{0.5\baselineskip}  % Abstand zwischen Absätzen
\rmfamily
\renewcommand*{\chapterpagestyle}{scrheadings}
\pagestyle{scrheadings}
\onehalfspacing
% \include{content goes here}
\chapter{Einleitung}
\autoref{sec:einleitung}
Die Robotik ist ein breites Forschungsfeld mit praktisch grenzenlosen Möglichkeiten.
Die Fähigkeit, sich zu bewegen, ist dabei besonders spannend, da sie die Flexibilität der Einsatzmöglichkeiten für Roboter stark erhöht.
Roboter mit Beinen stellen sich besonders heraus.
Mit Inspirationen aus Mensch- und Tierreich bieten diese das Potenzial, sich in jedem denkbaren Terrain fortzubewegen, das auch für Menschen zugänglich ist oder sogar in Gebiete vorzudringen, die uns verwehrt bleiben.
Im Vergleich zu anderen Fortbewegungsarten stellt die stabile Koordination von mehreren Beinen, die jeweils aus mehreren Gelenken bestehen, allerdings eine große Herausforderung für die technische Umsetzung dar.
In der Regel sind für die Programmierung solcher Roboter eine sehr genaue Kenntnis der Maschine und deren Dynamik vonnöten.

In den vergangenen Jahren wird deshalb verstärkt erforscht, wie Roboter sich diese Fähigkeiten selbstständig mittels Reinforcement Learning beibringen können.
In einer vorangegangenen Studienarbeit wurden am Beispiel eines eigens dafür gebauten, spinnenartigen, vierbeinigen Roboters Möglichkeiten erforscht, um diesen mittels selbst gelernter Bewegungsabläufe möglichst effizient und schnell geradlinig nach vorne zu bewegen.
Um den Trainingsprozess zu beschleunigen wurde dabei der Roboter in der Simulationsumgebung \enquote{Unity} nachgebaut und trainiert.

Für das Erfüllen eines praktischen Nutzens ist es jedoch in der Regel nicht ausreichend, wenn sich ein Roboter nur in eine feste Richtung bewegen kann.
Im Rahmen dieser Arbeit wird deshalb untersucht, wie dem Roboter beigebracht werden kann, einem gezielt übergebenen Pfad zu folgen.
Weiterhin soll der Roboter dabei Hindernisse, die sich auf diesem Pfad befinden, automatisch umsteuern und anschließend wieder auf den vorgebenen Pfad zurückkehren. % ggf Absatz hier
Dazu werden zunächst die Arbeitsumgebung und Ergebnisse der vorherigen Arbeit rekonstruiert.
Anschließend wird diskutiert, welche Änderungen am Roboter und dessen Simulationsumgebung vorgenommen werden müssen, um die erweiterten Anforderungen grundsätzlich erfüllen zu können.
Außerdem wird erläutert, wie die Aufgabe in sinnvolle Teilaufgaben gegliedert werden kann.
Zur Bearbeitung dieser Teilaufgaben wird dann ein Konzept erarbeitet, welches im Anschluss in der Simulationsumgebung, unter Zuhilfenahme des ML-Agents-Toolkits, mittels des \acl{ppo}-Algorithmus ein Proof-of-Concept trainiert.
Anhand der Ergebnisse dieses Trainingsprozesses wird evaluiert, ob das entwickelte Konzept einen erfolgversprechenden Lösungsansatz darstellt und welche Verbesserungen vorgenommen werden sollten.

% \begin{itemize}
%     \item Roboter sind ein breites Forschungsfeld mit grenzenlosen Möglichkeiten
%     \item Fähigkeit, sich zu bewegen ist besonders spannend für die Robotik, da es die Einsatzmöglichkeiten für Roboter stark erweitert
%     \item dabei insbesondere mehrbeinige Roboter interessant
%     \item mit Inspiration aus Menschen- und Tierreich sind diese fähig, sich in praktisch jedem denkbaren Terrain fortzubewegen
%     \item im Vergleich zu anderen Fortbewegungsarten stellt stabile Koordination von Beinen jedoch eine große Herausforderung für die technische Umsetzung voraus
%     \item in der Regel sehr genaue Kenntnis des Roboters und dessen Dynamik vonnöten
    
%     \item in den vergangenen Jahren wird deshalb verstärkt erforscht, wie Roboter sich diese Fähigkeit gegebenenfalls selbstständig mittels Reinforcement Learning beibringen können
%     \item in einer vorangegangenen Studienarbeit wurden am Beispiel eines eigens dafür gebauten, vierbeinigen Roboters Möglichkeiten erforscht, um diesen möglichst effizient und schnell geradlinig nach vorne zu bewegen
%     \item dabei wurde der Roboter in der Simulationsumgebung \enquote{Unity} trainiert
%     \item für das Erfüllen eines praktischen Nutzens ist es jedoch in der Regel nicht ausreichend, wenn sich ein Roboter nur in eine Richtung bewegen kann
%     \item im Rahmen dieser Arbeit wird deshalb untersucht, wie dem Roboter beigebracht werden kann, einem gezielt übergebenen Pfad zu folgen
%     \item weiterhin soll der Roboter Hindernisse, die sich auf diesem Pfad befinden, automatisch umsteuern und anschließend wieder auf den vorgegebenen Pfad zurückkehren
    
%     \item dazu werden zunächst die Ergebnisse und Arbeitsumgebung der vorherigen Arbeit rekonstruiert
%     \item anschließend wird diskutiert, welche Änderungen am Roboter und dessen Simulationsumgebung vorgenommen werden müssen, um die erweiterten Anforderungen grundsätzlich erfüllen zu können
%     \item außerdem wird erläutert, wie die Aufgabe in sinnvolle Teilaufgaben gegliedert werden kann
%     \item zur Bearbeitung dieser Teilaufgaben wird dann ein Konzept erarbeitet, welches im Anschluss in der Simulationsumgebung unter Zuhilfenahme des ML-Agents-Toolkits als Proof-of-Concept trainiert wird
%     \item Proximal-Policy-Optimization
% \end{itemize}

\chapter{Grundlagen}
\section{Machine Learning}
Machine Learning ist eine Unterkategorie der künstlichen Intelligenz und bezeichnet einen automatisierten Prozess, der es Computern ermöglicht, eigenständig aus Trainingsdaten zu lernen und sich mit der Zeit zu verbessern, ohne explizit zur Lösung einer Aufgabe programmiert zu werden.
Machine Learning Algorithmen können Muster in Daten entdecken und aus ihnen Lernen, um eigene Prognosen und Entscheidungen zu treffen.

In der Regel wird Machine Learning in folgende Teilbereiche untergliedert \cite{monkeylearnIntroductionMachine}:
\begin{itemize}
    \item \textbf{Supervised Learning:}
    Ein Modell wird anhand von gelabelten Trainingsdaten trainiert.
    Ein Datentupel besteht dabei aus einer Eingabe und der dazu gewollten Ausgabe.
    Der Algorithmus sucht beim Training nach Zusammenhängen und Abhängigkeiten um anschließend Ausgaben für unbekannte Eingaben generieren zu können.
    Üblicherweise wird Supervised Learning für Regressions- und Klassifikations-Probleme eingesetzt.

    \item \textbf{Unsupervised Learning:}
    Wird in der Regel für Clusterbildung von unbeschrifteten Trainingsdaten verwendet.
    Der Algorithmus muss dabei selbstständig nach Mustern in den Daten suchen.
    Unsupervised Learning kann dabei helfen, Einblicke in große Datensätzen zu erhalten, um etwa versteckte Trends zu entdecken.

    \item \textbf{Semi-Supervised Learning:}
    Für das Training werden beim Semi-Supervised Learning sowohl ein kleiner gelabelter, als auch ein großer ungelabelter Datensatz verwendet.
    Dabei werden die Vorzüge von Supervised und Unsupervised Learning miteinander verbunden.
    Interessant ist dies vor allem bei sehr großen Datensätzen (zum Beispiel bei Bild-Klassifizierung), da die Labelung der Daten in der Regel manuell erfolgen muss.

    \item \textbf{Reinforcement Learning:}
    Beim Reinforcement Learning kann ein Agent Aktionen tätigen, für die er entweder belohnt oder bestraft wird.
    Es ist sein Ziel, selbstständig ein bestmögliches Verhalten zu lernen, um seine Belohnung zu maximieren.
    Dabei werden keine Trainingsdaten verwendet.
    Der Agent lernt ausschließlich aus seinen eigenen Erfahrungen und Fehlern.
    Reinforcement Learning findet vor allem in den Bereichen Robotik und Videospiele Einsatz und wird auch in dieser Arbeit verwendet werden.
\end{itemize}

% \begin{itemize}
%     \item Unterkategorie der künstlichen Intelligenz
%     \item ML ist ein automatisierter Prozess, der ...
%     \item ermöglicht Computern, eigenständig aus Trainingsdaten zu lernen und sich mit der Zeit zu verbessern, ohne explizit programmiert zu werden
%     \item ML Algorithmen können Muster in Daten entdecken und aus ihnen lernen, um eigene Voraussagen und Entscheidungen zu treffen
    
%     \item verschiedene Teilbereiche
%     \item Supervised Learning
    
%     Vorhersagen von gelabelten Trainingsdaten
%     Datensatz: Input mit gewolltem Output
%     \item Unsupervised Learning
    
%     Einblicke und Beziehungen in unbeschrifteten Daten
%     Modelle müssen selbstständig Muster finden
%     Clustering
%     kann helfen, versteckte Muster oder Trends zu entdecken
%     \item Semi-Supervised Learning
    
%     kleiner Teil gelabelter Daten, viele ungelabelte
%     besonders für große Datensätze interessant, Bild-Klassifizierung
%     \item Reinforcement Learning
    
%     vor allem im Bereich Robotik und Videospiele vertreten
%     Agent kann Aktionen tätigen, für die er belohnt oder bestraft wird
%     es ist sein Ziel, ein bestmögliches Verhalten zu lernen, um seine Belohnung zu maximieren
%     keine Trainingsdaten, Lernen aus eigenen Fehlern
%     in diesem Kontext besonders interessant wird verwendet

%     \item \cite{monkeylearnIntroductionMachine}
% \end{itemize}

\subsection{Reinforcement Learning}
Reinforcement Learning ist ein rechnerischer Ansatz, um zielorientierte Lern- und Entscheidungsprozesse zu verstehen und nachzubilden.
Wie oben schon beschrieben, steht dabei ein \emph{Agent} im Zentrum, der aus der direkten Interaktion mit seiner \emph{Umgebung (engl. Environment)} lernt, ohne dabei eine beispielhafte Anleitung oder vollständige Modelle der Umgebung zu benötigen.
Das formale Framework des \emph{\ac{mdp}} wird genutzt, um die Interaktion zwischen dem lernenden Agenten und seiner Umgebung zu definieren.
\acp{mdp} sind eine mathematisch idealisierte Form eines Reinforcement Learning Problems, für die präzise, theoretische Aussagen getroffen werden können \cite[13]{sutton2018rlintro}.
\autoref{fig:mdp} stellt die grundlegende Struktur eines Reinforcement Learning Problems dar.
Die einzelnen Bestandteile der Abbildung werden nachfolgend erläutert.

\begin{figure}
    \centering
    \includegraphics[width = 0.7\textwidth]{Bilder/MDP.pdf}
    \caption{Die Interaktion von Agent und Umgebung als \ac{mdp} \cite[48]{sutton2018rlintro}}
    \label{fig:mdp}
\end{figure}

Agenten haben explizite Ziele, können Aspekte ihrer Umgebung wahrnehmen und \emph{Aktionen} $A$ auswählen, um mit ihrer Umgebung zu interagieren und diese zu beeinflussen.
Es wird davon ausgegangen, dass Reinforcement Learning diejenige Strategie des Machine Learning ist, die dem natürlichen Lernen von Menschen und Tieren am nächsten kommt.
Viele zentrale Algorithmen des Reinforcement Learnings sind ursprünglich durch biologische Systeme inspiriert \cite[4]{sutton2018rlintro}.

Besonders wichtig für Reinforcement Learning ist das Konzept von \emph{Zuständen} $S$.
Ein Zustand kann dabei als eine Art Signal verstanden werden, das dem Agent Informationen über den Zustand der Umgebung liefert.
Weiterhin definieren folgende Elemente ein Reinforcement Learning Problem:
\begin{itemize}
    \item \textbf{Policy} $\mathbf{\pi}$\textbf{:}
    Die Policy definiert, wie sich der Agent zu einer gegebenen Zeit verhält.
    Sie stellt ein Mapping zwischen den wahrgenommenen Zuständen der Umgebung und den durchzuführenden Aktionen dar.
    Die Policy ist hinreichend, um das Verhalten des Agent zu bestimmen \cite[6]{sutton2018rlintro}.

    \item \textbf{Reward-Signal} $\mathbf{R}$\textbf{:}
    Das Reward-Signal definiert das Ziel eines Reinforcement Learning Problems.
    Bei jedem \emph{Zeitschritt} $t$ sendet die Umgebung ein Skalar an den Agent.
    Das einzige Ziel des Agent ist die Maximierung des kumulativen Rewards.
    Der Reward ist die primäre Basis für Änderungen an der Policy \cite[6]{sutton2018rlintro}.

    \item \textbf{Value-Funktion} $\mathbf{V}$\textbf{:}
    Die Value-Funktion legt fest, was auf lange Sicht gut ist.
    Der Value eines Zustands ist der kumulierte Reward, den ein Agent, ausgehend von diesem Zustand, in der Zukunft erwarten kann.
    Values geben die langfristige Attraktivität von Zuständen an.
    Die Wahl einer Aktion wird auf Basis der Value-Einschätzung des aktuellen Zustands getroffen.
    Im Vergleich zum Reward-Signal sind Values allerdings deutlich schwerer zu bestimmen, da diese anhand einer Sequenz von Observationen des Agenten geschätzt werden müssen \cite[6]{sutton2018rlintro}.

    \item \textbf{Modell der Umgebung (optional):}
    Manche Reinforcement Learning Systeme nutzen ein Modell der Umgebung.
    Dieses Modell erlaubt das Ziehen von Schlussfolgerungen über das Verhalten der Umgebung.
    So kann etwa eine Voraussage des nächsten Resultierenden Zustands und Rewards, ausgehend von einem gegebenen Zustand und einer Aktion getroffen werden.
    Genutzt werden diese Modelle zur Planung.
    Es werden also Entscheidungen für eine Folge von Aktionen auf Basis möglicher zukünftiger Situationen getroffen, bevor diese tatsächlich erlebt werden.
    Reinforcement Learning Methoden, die Modelle und Planung verwenden, werden als \emph{modell-basiert} bezeichnet.
Im Gegensatz dazu stehen \emph{modell-freie} Methoden, welche explizit auf Basis von Trial-And-Error lernen \cite[7]{sutton2018rlintro}.
\end{itemize}

Beim Reinforcement Learning versucht der Agent, mit seinen ausgeführten Aktionen ein Reward-Signal zu maximieren.
Dabei muss ein Kompromiss zwischen Nutzen des Gelerntem und Entdecken von Neuem gefunden werden.
Es besteht das Dilemma, dass weder das eine, noch das andere uneingeschränkt verfolgt werden kann, ohne bei der Ausführung der Aufgabe zu scheitern, denn beim Entdecken muss der Agent auch schlechte Aktionen ausführen, für die er keinen Reward erhält.
Entdeckt er jedoch nichts, weiß er auch nicht, welche Aktionen einen hohen Reward erzeugen \cite[3]{sutton2018rlintro}.

% \begin{itemize}
%     \item Markov Decision Processes sind eine mathematisch idealisierte Form eines Reinforcement Learning Problems für die präzise theoretische Aussagen getroffen werden können
    
%     \item rechnerischer Ansatz, um zielorientierte Lern- und Entscheidungsprozesse zu verstehen
%     \item im Zentrum steht dabei ein Agent, der aus der direkten Interaktion mit seiner Umgebung lernt, ohne dabei eine beispielhafte Anleitung oder vollständige Modelle der Umgebung zu benötigen
%     \item nutzt das formale Framework des Markov Decision Processes um die Interaktion zwischen dem lernenden Agenten und seiner Umgebung zu definieren
%     \item Versuch, ein Reward-Signal zu maximieren
%     \item Kompromiss zwischen Nutzen und Entdecken
%     \item Dilemma: weder das eine, noch das andere kann uneingeschränkt verfolgt werden, ohne zu scheitern
%     \item Agenten haben explizite Ziele, können Aspekte ihrer Umgebung wahrnehmen und Aktionen auswählen, um ihre Umgebung zu beeinflussen
%     \item viele zentrale Algorithmen des Reinforcement Learning sind ursprünglich durch biologische Systeme beeinflusst
    
%     \item Konzept von Zuständen; ein Zustand ist eine Art Signal, das dem Agent Informationen über den Zustand der Umgebung liefert
    
%     Elemente:
%     \item Policy
    
%     definiert, wie sich der Agent zu einer gegebenen Zeit verhält
%     Mapping zwischen wahrgenommenen Zuständen der Umgebung und durchzuführenden Aktionen
%     hinreichend um das Verhalten zu bestimmen
%     \item Reward-Signal
    
%     definiert das Ziel eines Reinforcement Learning Problems
%     bei jedem Zeitschritt sendet die Umgebung eine einzelne Nummer an den Agent
%     das einzige Ziel des Agents ist die Maximierung des kommulativen Rewards
%     primäre Basis für Änderungen an der Policy
%     \item Value-Funktion
    
%     legt fest, was auf lange Sicht gut ist
%     Value eines Zustands ist der kommulierte Reward, den ein Agent, ausgehend von diesem Zustand, in der Zukunft erwarten kann
%     Values geben die langfristige Attraktivität eines Zustands an
%     die Wahl einer Aktion wird auf Basis der Value-Einschätzung getroffen
%     aber, Values sind deutlich schwerer zu bestimmen als Rewards
%     \item optional Modell der Umgebung
    
%     manche Reinforcement Learning Systeme nutzen ein Modell der Umgebung
%     erlaubt das Ziehen von Schlussfolgerungen über das Verhalten der Umgebung
%     etwa: Voraussagen des nächsten resultierenden Zustands und Rewards ausgehend von einem Zustand und einer Aktion
%     werden zur Planung genutzt, also Entscheiden für eine Folge von Aktionen auf Basis möglicher zukünftiger Situationen bevor diese tatsächlich erlebt werden
%     Reinforcement Learning Methoden, die Modelle und Planung verwenden, werden als modell-basiert bezeichnet
%     im Gegensatz dazu stehen modell-freie Methoden, also explizites Lernen auf Basis von trial-and-error 
% \end{itemize}
% Begriffsdefinitionen
% Markov Decision Process


\subsection{Deep Reinforcement Learning}
Wie auch andere Algorithmen haben Reinforcement Learning Algorithmen Skalierungsprobleme hinsichtlich ihrer Komplexität.
So kommt es beispielsweise zu Schwierigkeiten, die Value- oder die Policy-Funktion abzubilden, wenn die Dimension des Reinforcement Learning Problems zu groß wird.
Insbesondere bei hoch-dimensionalen, kontinuierlichen Zustands- und Aktionsräumen ist dies der Fall.

Die wichtigste Eigenschaft von Deep Learning sind Deep Neural Networks.
Diese Netzwerke können automatisch kompakte, niedrig-dimensionale Repräsentationen von hoch-dimensionalen Daten finden.
Beim Deep Reinforcement Learning werden Algorithmen und Technologien des Deep Learning in Reinforcement Learning eingebracht.
Dabei werden Deep Neural Networks als Funktionsapproximatoren für die Value-Funktion oder die Policy verwendet.
Diese neue Kombination macht eine Skalierung auf bislang unlösbare Entscheidungsprobleme möglich \cite{Arulkumaran2017}.

% \begin{itemize}
%     \item RL Algorithmen, wie andere Algorithmen auch haben Problem mit Komplexität: Speicherkomplexität, Rechenkomplexität und Probenkomplexität (letztere ML spezifisch)
%     \item Deep Learning, basierend auf den mächtigen Funktionsapproximationen und repräsentativen Lerneigenschaften von Deep Neural Networks, bietet neue Tools, um gegen diese Probleme anzukommen
%     \item wichtigste Eigenschaft von Deep Learning: Deep Neural Networks können automatisch kompakte, niedrig-dimensionale Repräsentationen von hoch-dimensionalen Daten finden
%     \item grundsätzlich eigentlich nur Deep Learning Algorithmen im Kontext von RL
%     \item macht Skalierung auf bislang unlösbare Entscheidungsprobleme möglich (z.B. hoch-dimensionale Zustands- und Aktionsräume)
%     \item 
% \end{itemize}

\section{Unity3D}
Unity3D ist eine plattformübergreifende Game Engine, die erstmal 2005 angekündigt wurde.
Primärer Zweck der Unity Engine ist die Entwicklung von Videospielen für Computer, Konsolen und Mobilgeräte.
Dabei ist Unterstützung für zwei- und dreidimensionale Grafik enthalten.
VR Entwicklung ist ebenso möglich.\todo{Abkürzungen}
Das Skripting innerhalb der Engine erfolgt primär in C\# \cite{freecodecamp.unityIntroduction}.
Neben dem Einsatz in der Spieleentwicklung ist Unity jedoch auch für den Einsatz in anderen Branchen geeignet, so zum Beispiel in der Architektur oder der Forschung \cite[30]{waidner.2020}, wo mit Unity Simulationen der realen Welt erstellt werden können.
Als Grafik-APIs werden unter anderem Direct3D (Windows), OpenGL (Linux, macOS, Windows) und WebGL unterstützt.
Unity enthält einen Asset Store für die Entwickler-Community, über den Dritten das Hoch- und Herunterladen kommerzieller und freier Ressourcen (zum Beispiel Texturen, Modelle und Plugins) ermöglicht wird \cite{freecodecamp.unityIntroduction}.
Der Einsatz von Unity für Projekte mit weniger als 100.000 \$ jährlichem Gewinn ist kostenlos \cite{unityPersonal}.

% cross-plattform Game Engine
% 2005 angekündigt
% primär für die Entwicklung von Videospielen und Simulationen für Computer, Konsolen und Mobilgeräten
% unterstützt 2D und 3D Grafik und C\# Skripting
% auch für VR Entwicklung geeignet
% Grafik APIs: unter anderem Direct3D (Windows), OpenGL (Linux, macOS, Windows), WebGL
% Asset Store für Entwickler Community, Upload \& Download für kommerzielle und freie Ressourcen Dritter (Texturen, Modelle, Plugins)
% \cite{freecodecamp.unityIntroduction}
% wird auch außerhalb des Spielebereichs verwendet, so auch für Architektur und Forschung \cite[30]{waidner.2020}
% für nicht wirtschaftliche Projekte kostenlos (Quelle Unity Projektseite)

\subsection{Unity Machine Learning Agents Toolkit}
Das Unity Machine Learning Agents Toolkit (kurz \emph{ML-Agents}) ist ein von Unity Technologies entwickeltes, quelloffenes Projekt, welches 2017 erstmals als Testversion veröffentlicht wurde, seitdem sehr aktiv weiterentwickelt wird und inzwischen die Produktreife erreicht hat.
ML-Agents ermöglicht es Spiele und Simulationen, als Trainingsumgebung für intelligente Agenten zu dienen.
Dabei werden State-of-the-Art, PyTorch-basierte Implementierungen gängiger Machine Learning Algorithmen angeboten, um ein einfaches Training mit möglichst geringer Einstiegshürde zu ermöglichen.
Alternativ können auch eigene Algorithmen zum Training verwendet werden.
Wie Unity selbst auch ist ML-Agents für den Einsatz in 2D-, 3D- und VR/AR-Umgebungen geeignet.
Als Trainingsmethoden werden unter anderem Reinforcement Learning, Imitation Learning und Neuroevolution unterstützt \cite{mlagentsDocHome}.

% - quelloffenes Projekt, 2017 veröffentlicht
% - ermöglicht Spielen und Simulationen, als Trainingsumgebung für intelligente Agents zu dienen
% - bietet state-of-the-art, PyTorch-basierte Implementierungen gängiger Machine Learning Algorithmen, um ein einfaches Training mit möglichst geringer Einstiegshürde zu ermöglichen
% - es können auch eigene Algorithmen zum Training verwendet werden
% - unterstützt werden auch hier 2D, 3D und VR/AR
% - als Methoden für das Training werden unter anderem Reinforcement Learning, Imitation Learning und Neuroevolution unterstützt
% \todo{ausformulieren}
% \cite{mlagentsDocHome}

ML-Agents besteht aus folgenden high-level Komponenten \cite{mlagentsOverview} (siehe \autoref{fig:learning-environment-basic}):
\begin{itemize}
    \item \textbf{Trainingsumgebung (Learning Environment):}
    Die Trainingsumgebung enthält eine Unity Szene und sämtliche Game Charaktere.
    Die Unity Szene stellt dabei die Umgebung bereit, in der Agenten ihre Beobachtungen machen, handeln und lernen.
    Mithilfe des ML-Agents Unity SDK kann jede Unity Szene in eine Trainingsumgebung transformiert werden, indem Game Objects als Agenten definiert werden.

    \item \textbf{Python Low-Level API:}
    Die Python API enthält ein low-level Python Interface, welches die Aufgabe besitzt, mit der Trainingsumgebung zu interagieren und diese zu manipulieren.
    Diese Python API ist im Gegensatz zur Trainingsumgebung kein Teil von Unity, sondern kommuniziert mit Unity durch den Communicator.

    \item \textbf{Externer Communicator:}
    Der Communicator erfüllt die Aufgabe, die Python API mit der Trainingsumgebung zu verbinden.

    \item \textbf{Python Trainer:}
    In den Python Trainern sind alle Machine Learning Algorithmen enthalten, die ein Training der Agenten ermöglicht.
    Dieses Paket stellt die zum Training genutzte CLI\todo{Abkürzung} (\code{mlagents-learn}) bereit.
\end{itemize}

\begin{figure}
    \centering
    \includegraphics[width = 0.5\textwidth]{Bilder/ml-agents/learning_environment_basic.png}
    \caption{Vereinfachtes Block-Diagramm des ML-Agents-Toolkits \cite{mlagentsOverview}}
    \label{fig:learning-environment-basic}
\end{figure}

Die Trainingsumgebung wird durch zwei enthaltene Unity-Komponenten organisiert \cite{mlagentsOverview}.
\begin{itemize}
    \item \textbf{Agents} sind an ein Unity GameObject geknüpft (beliebiger Charakter innerhalb einer Szene).
    Sie sind gleichzusetzen mit dem Agent eines Reinforcement Learning Problems.
    Agents generieren die Observations (Beobachtungen) des GameObjects, welche dem Reinforcement Learning Algorithmus zugeführt werden, führen die vom Algorithmus empfangenen Aktionen aus und weisen den Reward zu.
    Jeder Agent ist mit einem Behavior verknüpft.

    \item \textbf{Behaviors} (Verhalten) definieren Attribute des Agenten, so auch die Anzahl der Aktionen, die der Agent entgegennehmen kann.
    Ein Behavior kann als Funktion verstanden werden, welche Observations und Reward des Agents als Eingabeparameter enthält und auszuführende Aktionen zurückliefert.
    Behaviors werden in drei Typen unterschieden: \emph{Learning}, \emph{Heuristic} und \emph{Inference}.
    Learning Behaviors sind noch nicht definiert, können aber trainiert werden.
    Heurisitc Behaviors werden mittels manuell implementierter Regeln im Quellcode definiert.
    Inference Behavior werden von trainierten Neural-Network-Dateien (entsprechen der finalen, trainierten Policy) repräsentiert.
    Nachdem ein Learning Behavior trainiert wurde, wird es zum Inference Behavior.
\end{itemize}
Herausstellenswert hierbei ist, dass ML-Agents die Möglichkeit bietet, mehrere Agents in einer Trainingsumgebung zu platzieren, die jedoch mit demselben Behavior verknüpft sein können.
Dies kann dafür genutzt werden, das Training zu parallelisieren und damit zu beschleunigen.

\subsection{Reinforcement Learning Algorithmen}
Von ML-Agents werden zwei Trainingsalgorithmen bereitgestellt, die sich dem (Deep) Reinforcement Learning zuordnen lassen.
Dies sind \acf{ppo} und \acf{sac} \cite{mlagentsOverview}.
In \cite{waidner.2020} wurden diese Algorithmen bereits verglichen.
\ac{ppo} ist der Standardalgorithmus von ML-Agents, da er sich, verglichen mit vielen anderen Reinforcement Learning Algorithmen, als für den allgemeinen Einsatz besser geeignet gezeigt hat \cite{schulman2017proximal,openaiPPO}.
\ac{ppo} ist ein on-policy Algorithmus.
Das bedeutet, dass in jeder Iteration des Lernvorgangs nur aus Erfahrungen gelernt wird, die mit der aktuellen Version der Policy gesammelt wurden.
\ac{sac} hingegen ist ein off-policy Algorithmus und lernt somit aus allen Erfahrungen, die er jemals während des gesamten Trainingsvorgangs gesammelt hat \cite{suran2020,sagar2020}.
Daraus ergeben sich für beide Algorithmen unterschiedliche Vor- und Nachteile.
On-policy Algorithmen haben in der Regel einen deutlich stabileren Lernfortschritt, als off-policy Algorithmen.
Andererseits brauchen on-policy Algorithmen in der Regel deutlich mehr Trainingsschritte, um nennenswerte Ergebnisse zu erzielen \cite{mlagentsOverview}.
Im Zuge der Vorgängerarbeit wurde mit beiden Algorithmen gearbeitet, mit dem Ergebnis, dass der \ac{ppo}-Algorithmus auch für das konkrete Problem im Rahmen dieser Studienarbeit deutlich bessere Resultate liefert \cite[48]{waidner.2020}.

\subsection{Hyperparameter}


\subsection{Bewertungskriterien?}
Value Loss
stetiges Training
\cite{aurelian2018,untiyMetrics}

\section{Beschreibung der Projektbasis}

\section{Vektorgeometrie}
Abstandsbestimmung und Winkel zwischen Vektoren
\chapter{State of the Art}


\enquote{Decentralized Deep Reinforcement Learning for a Distributed and
Adaptive Locomotion Controller of a Hexapod Robot}
Machine Learning in letzten Jahren erfolgreich auf viele Aufgaben angewandt
DRL scheint noch Probleme zu haben bei der Anwendung für reale Roboter in \enquote{continuous control tasks}
vor allem im Umgang mit unvorhergesehenen Situationen gibt es Probleme

ursprünglich aus Bereich Computerspiele, deshalb viel in simulierten Umgebungen
Transfer auf reale Probleme kann schwierig sein (\enquote{nature of such problems is fundamentally different from those in playing computer games})
häufig wird zunächst die Simulation genutzt, um Grundsteine zu legen, die dann manuell feingeschliffen werden für eine bestimmte Aufgabe
zwei fundamentale Probleme: soll den Reward ausnutzen, neigt deshalb zu Overfitting; reale Anwendungen deutlich mehr (Signal-)Rauschen, führt zu Hinterfragen von festem Markov Decision Process
DRL tendiert dazu, Nischenlösungen zu finden, die meist nicht dazu in der Lage sind, adaptiv auf neue Situationen zu reagieren

Tendenz geht dahin, hierarchische oder dezentralisierte Ansätze zu verfolgen
hierarchisch erlaubt flexibles wechseln zwischen verschiedenen Unteraufgaben und Verhaltensweisen und somit auch Agieren in verschiedenen Kontexten möglich, bislang allerdings nur mit geringen Freiheitsgraden umgesetzt
Fokus diesen Papers eher auf Störungen und Varietät in einem spezifischen Kontext mit einem spezifischen Verhalten

PPO funktioniert allgemein gut mit kontinuierlichen Problemen ohne viel Hyperparameter-Tuning
Median-Geschwindigkeit einer Episode ist der Reward
\cite{schilling2020decentralized}



\enquote{Adaptation of a Decentralized Controller to Curve Walking in a Hexapod Robot}
bei Robotern mit mehreren Beinen aktuell drei vorherrschende Ansätze
- Central Pattern Generators (CPGs)
    - Oszilatorsysteme oder neurale Netze, die rhythmische Ausgabe erzeugen, ohne bestimmte Eingabe vorauszusetzen
- lernende Ansätze
- Sensorenbasierte Ansätze
    - sind besser erklärbar verglichen mit lernenden Ansätzen
    - normalerweise relativ anpassbar an sich verändernde Umgebungen

sehen Potenzial in Verbindung mehrerer Ansätze
\cite{simmering2023walknet}



\enquote{DeepGait: Planning and Control of Quadrupedal Gaits using Deep Reinforcement Learning}
Kombination von state-of-the-art modellbasierten Methoden der Bewegungsplanung und Reinforcement Learning
Evaluieren die physikalische Machbarkeit anstatt physisch zu simulieren
trennen Schrittplanung und Ausführung
es können ganze Schritte evaluiert werden und nicht nur einzelne Frames einer Simulation
Hauptproblem in Schrittplanung ist Kombinatorik, wegen der vielen möglichen Kombinationsmöglichkeiten für Kontaktpunkte mit dem Untergrund
aber: Training auf späterem Terrain, allerdings relativ gute Verallgemeinerung
Entropie ist extrem wichtig
Weit bessere Resultate als andere Ansätze, gerade was das Überbrücken von Klüften angeht
\cite{tsounis2020deepgait}



\enquote{Learning and Adapting Agile Locomotion Skills by Transferring Experience}
selbst einfache Aufgaben können sehr komplexe modulierte Reward-Funktionen benötigen, um die gezielten Bewegungen zu erhalten
Wenn eine Bewegungsform sehr gut koordinierte Bewegungen voraussetzt, kann sehr schwer sein, wenn man von Null beginnt (-> nur sehr konkretes Verhalten kann Reward nach sich ziehen)
Für ein spezifisches Zielproblem zu trainieren mag schwer sein, doch häufig ist es möglich, mit einfacheren Trainingsumgebungen zumindest relevante Daten zu erhalten, welche für einen Lernprozess von Interesse sind
(Aus auf Hinterbeinen stehen wird Laufen)
Häufig ist es schwierig, das in agileres Verhalten zu erweitern; insbesondere mit stark angepassten Reward-Funktionen, um das Verhalten überhaupt erst zu erzeugen
=> Training Agiler Roboter Fähigkeiten erleichtern durch Transfer mit existierenden suboptimalen Fähigkeiten
\cite{smith2023learning}

\begin{figure}
    \centering
    \includegraphics[width = \textwidth]{Bilder/transfer-learning.pdf}
    \caption{\cite{smith2023learning}}
\end{figure}
\chapter{Konzeptionierung}
\section{Einschränkungen und Übertragungsprobleme}
\label{sec:probleme}
In der Vorgängerarbeit sollte der Roboter bislang nur geradeaus laufen.
Dabei verfügt der verwendete Roboter über keinerlei Sensorik.
Einzig die aktuellen Zielwinkel der Servomotoren, welche die Bewegung der Beine ermöglichen, sind dem Lernalgorithmus zugänglich.
Das Training des Roboters erfolgte jedoch rein in der Simulation.
Da die Simulationsumgebung nicht nur den Roboter, sondern dessen komplettes Umfeld abbilden muss, sind alle Informationen, die ein beliebiger Sensor liefern kann, theoretisch in der Simulation vorhanden, wurden dem Algorithmus jedoch nicht zugänglich gemacht.
Der Roboter kennt seinen eigenen Zustand nicht, beziehungsweise nur bedingt.
Das trainierte Modell wird lediglich in der Praxis angewandt, unter der Annahme, dass bei einem korrekt gelernten Modell keinerlei Zusatzinformationen notwendig sind, um den Roboter seine Aufgabe erfüllen zu lassen: geradeaus zu laufen.

Die Zielsetzung dieser Studienarbeit erweitert nun jedoch diese Aufgabe des Roboters, was Probleme aufwirft.
Der Roboter soll lernen, einem vorgegebenen Pfad zu folgen.
Dabei soll der Pfad für jeden Durchlauf dem Roboter individuell vorgegeben werden können.
Besonders wichtig ist hierbei zu beachten, dass der Roboter unter keinen Umständen einen bestimmten Pfad auswendig lernen soll, denn dann muss für jeden Pfad, den der Roboter laufen soll, ein eigenständiges Modell trainiert werden, was einen praktischen Nutzen unmöglich macht -- schließlich kann nicht für eine Bewegungsanweisung an einen Roboter jedes Mal mehrere Stunden Rechenzeit aufgebracht werden.
Dem Roboter muss also ein Pfad mitgegeben werden.
Außerdem muss für das Training des Roboters mehrfach dieser übergebene Pfad neu generiert werden, um ein Auswendiglernen zu verhindern.

Dass der Roboter einem frei angegebenen Pfad folgen können soll, wirft die Frage auf, ob er dazu Informationen über seine Position im Raum benötigt.
Rein theoretisch betrachtet kann diese Frage einfach mit \enquote{Nein} beantwortet werden.
Prinzipiell gesehen kann der Roboter seine aktuelle Position anhand seiner vergangenen Bewegungen vom Ausgangspunkt her berechnen.
Andererseits setzt dies eine enorm hohe Präzision der Bewegungen voraus.
Außerdem kann der Roboter in der Realität auf dem Boden rutschen.
Auch läuft das bisherige Modell nicht verlässlich geradeaus -- nicht einmal in der Simulation --, sondern hat dabei immer einen leichten Drall zur Seite.
Aus diesen Gründen wird der Schluss gezogen, dass durch minimale, nicht vermeidbare Abweichungen die Ausführung der Aufgabe nur sehr ungenau möglich ist, wenn keine Positionierungsinformationen zur Verfügung gestellt werden.

In der Simulation gestaltet sich eine mögliche Lösung des Problems sehr einfach: Der Roboter ist ein GameObject innerhalb der Unity-Engine.
Als solches besitzt er automatisch Koordinaten innerhalb der Simulation, welche einfach für den Roboter freigegeben werden können.
Bei einer späteren Übertragung in die Realität können diese Informationen durch andere, konkrete Ortungssysteme geliefert werden.
Es ist lediglich ein Mapping erforderlich, um das Informationsformat eines konkreten Sensors in das Koordinatenformat von Unity umzuwandeln.
Diese Informationen können dann dem Modell für die Inferenz zur Verfügung gestellt werden.
Somit ist es möglich, ein Modell zu trainieren, welches unabhängig von der später eingesetzten Ortungstechnologie ist.

Weiterhin umfasst die Aufgabenstellung, dass der Roboter Hindernisse auf seinem Weg erkennen und gezielt umgehen können soll.
Anschließend soll er auf den vorgegebenen Pfad zurückkehren, wobei die Abweichung möglichst gering ausfallen sollte.
Hierfür wird eine Hindernis- beziehungsweise Kollisionserkennung benötigt.
Mit der aktuellen technischen Ausstattung des Roboters ist auch diese Aufgabe nicht umsetzbar.
In der Simulation soll vereinfacht für die Hinderniserkennung die Kollisionserkennung der Beine verwendet werden.
(Diese Kollisionserkennung kann Unity für sämtliche GameObjects durchführen.)
Dadurch muss der Roboter mit seinem Hindernis zusammenstoßen, um es wahrzunehmen.
In der Realität ist natürlich eine Hinderniserkennung sinnvoll, mit der Hindernisse bereits vor einer Kollision erkannt werden können (zum Beispiel LiDAR, Kameras oder ähnliche Systeme).

Sowohl für die Ortung, als auch für die Hinderniserkennung ist in der Realität der Einsatz komplexer Systeme nötigt.
Die Integration solcher übersteigt jedoch den Umfang dieser Arbeit erheblich.
Hier soll eher ein Proof of Concept für die selbstständige Umsteuerung von Hindernissen erarbeitet werden.
Es soll daher keine Übertragung der in der Simulation trainierten Modelle auf den realen Roboter stattfinden.

Die Springbewegung, die der Roboter sich bislang antrainiert hat, führt zu weiteren potenziellen Problemen.
Zwar wurde, wie oben beschrieben, der Einsatz eines Ortungssystems festgelegt, jedoch bringt diese Bewegungsform trotzdem massive Genauigkeitsprobleme mit sich. 
Sie ist sehr kraftaufwändig und instabil, der Roboter schwankt dabei stark um seine horizontale Achse und kann seine exakte Landeposition nur bedingt steuern.
Da der Roboter Hindernisse über eine Kollision mit diesen erkennen soll, besteht außerdem das Problem, dass der Roboter im Sprung gegen diese knallen kann.
Aus den genannten Gründen ist es sinnvoll, Maßnahmen zur Einschränkung der Bewegung des Roboters vorzunehmen.
Möglich ist zum Beispiel eine Anpassung der Reward-Funktion, wonach starke Höhenänderungen oder Neigungen der Zentralplatte des Roboters bestraft werden.

% \begin{itemize}
%     \item bislang sollte der Roboter nur geradeaus laufen
%     \item Roboter verfügt über keinerlei Sensorik
%     \item kennt nur den aktuellen Winkel der Beine/Servomotoren
%     \item Training erfolgte rein in der Simulation
%     \item die Simulationsumgebung kennt den Zustand (z.B. Neigung) des Roboters und kann anhand dessen den Wert der Belohnungsfunktion berechnen und an den Roboter zurückmelden
%     \item Roboter kennt seinen eigenen Zustand nicht
%     \item bisheriges Modell wird lediglich in der Praxis angewandt, unter der Annahme, dass bei korrekt gelerntem Modell keinerlei Zusatzinformationen notwendig sind, um den Roboter seine Aufgabe erfüllen zu lassen: geradeaus zu laufen
% \end{itemize}

% Probleme in der erweiterten Aufgabenstellung:
% \begin{itemize}
%     \item der Roboter soll nun einem vorgegebenen Pfad folgen
%     \item dabei soll der Pfad für jeden Durchlauf dem Roboter individuell vorgegeben werden können
%     \item der Roboter soll NICHT einen vorgegebenen Pfad lernen und danach immer von diesem Pfad ausgehen
%     \item dem Roboter muss also ein Pfad mitgegeben werden können
    
%     \item Positionierungsinformationen benötigt?
%     \item einerseits nein, theoretisch kann der Roboter seine aktuelle Position anhand seiner vergangenen Bewegungen vom Ausgangspunkt bestimmen
%     \item andererseits ja, da der Roboter auf dem Boden rutschen kann (zumindest in der Realität), das Berechnen von Entfernungen den gesamten Algorithmus stark verkompliziert und unter Umständen durch kleinere Abweichungen sehr ungenau ausfallen kann
%     \item mögliche Lösung in der Simulation: Positionierungsinformationen innerhalb der Simulationsumgebung für Roboter freigeben
%     \item diese Informationen könnten dem Roboter später durch andere Sensoren geliefert werden
%     \item wenn das Informationsformat des neuen Sensors umgewandelt wird in das bisherige Format (zum Beispiel durch ein externes Modul), könnte ein anderer Sensor Plug-And-Play in das trainierte Modell integriert werden
    
%     \item außerdem soll der Roboter Hindernisse auf seinem Weg erkennen und gezielt umgehen können
%     \item danach soll auf den Pfad zurückgekehrt werden
%     \item dafür wird eine Hindernis-/Kollisionserkennung benötigt
%     \item aktuell hat der Roboter keinerlei solche Sensorik
%     \item vereinfacht soll für die Hinderniserkennung in der Simulation die Kollisionserkennung für die Beine verwendet werden
%     \item dadurch muss der Roboter mit seinem Hindernis zusammenstoßen, um es wahrzunehmen
%     \item in der Realität wäre natürlich eine Hinderniserkennung sinnvoll, mit der Hindernisse bereits vor einer Kollision erkannt werden können (z.B. LiDAR, Kameras oder ähnliche), die Integration solcher Systeme würde jedoch den Umfang dieser Arbeit erheblich übersteigen
%     \item hier soll eher ein Proof-of-Concept für die selbstständige Umsteuerung von Hindernissen erarbeitet werden
    
%     \item Genauigkeitsprobleme beim Übertragen der Springbewegung: daher genauere Einschränkungen für den Algorithmus
%     \item bisher bewegt sich der Roboter nach Training in einer springenden Bewegung fort
%     \item diese Bewegungsform ist sehr kraftaufwändig und instabil, der Roboter schwankt dabei stark um die horizontale Achse und kann seine exakte Landeposition nur bedingt steuern
%     \item vor allem relevant, falls dem Roboter keine Positionierungsinformationen zur Verfügung gestellt würden, da dann jeder Millimeter zählt
%     \item aber auch zur Erhöhung der allgemeinen Genauigkeit und Reduzierung der Fehler, wäre es sinnvoll, die Fortbewegung des Roboters zu stabilisieren
%     \item mögliche Maßnahme zur Einschränkung der Bewegung: restriktive Anpassung der Belohnungsfunktion, wenn sich die Höhe des Mittelteils des Roboters zu stark verändert oder dieser spürbar die Neigung zum Horizont verändert, wird der Agent bestraft
    
%     \item Insgesamt soll keine Übertragung des implementierten Modells auf den realen Roboter stattfinden
%     \item wie oben beschrieben wäre eine Implementierung von diversen Sensoren vonnöten, was den Umfang dieser Arbeit deutlich übersteigt
% \end{itemize}

\section{Wahl der Simulationsumgebung}
In der Vorgängerarbeit wurden bereits drei mögliche Simulationsumgebungen ausführlich gegenübergestellt \cite[27]{waidner.2020}.
Die beschriebenen Bedingungen haben sich dabei leicht verändert.
Aus dem Vergleich ging hervor, dass MuJoCo aufgrund seiner äußert realistischen Simulation der Physik und einem Fokus auf Gelenksimulation eine gute Wahl für eine Trainingsumgebung ist.
Jedoch war der Einsatz von MuJoCo damals mit erheblichen Lizenzgebühren verbunden, was einer der Gründe war, diese Software nicht zu verwenden.
Mittlerweile ist MuJoCo allerdings frei und quelloffen verfügbar \cite{mujoco.org,github.mujoco}, was dafür sprechen würde, die Programmwahl neu zu überdenken.

Andererseits soll diese Arbeit an die vorangegangene anknüpfen.
Wenn die Simulationsumgebung gewechselt wird, muss im Grunde genommen von vorne begonnen werden, da die meisten Aspekte der Vorarbeit, wenn nicht sogar alle, nicht einfach außerhalb der Unity-Umgebung genutzt werden können.
Deshalb wird die Wahl getroffen, weiterhin Unity zu verwenden.
Unity bietet jedoch die Möglichkeit, die standardmäßig verwendete Physics-Engine gegen andere, über Plugins bereitgestellte, auszutauschen.
MuJoCo ist ebenfalls als ein solches Plugin verfügbar \cite{mujocoUnityPlugin}.
Insofern kann bei Bedarf theoretisch mit geringem Aufwand in Betracht gezogen werden, MuJoCo als Physics-Engine in Unity einzubinden, um somit das Gesamtergebnis des Trainings durch eine bessere Physiksimulation zu verbessern.
Auch ist es denkbar, die Ergebnisse der verschiedenen Umgebungen zu vergleichen.
Da jedoch keine Übertragung auf einen realen Roboter erfolgt, ist davon auszugehen, in diesem Kontext keinen spürbareren Unterschied zwischen den Physics-Engines zu beobachten.
Selbst falls ein messbarer Unterschied bestehen sollte, kann dieser nicht hinsichtlich seiner Aussagekraft eingeordnet werden.


% \begin{itemize}
%     \item in der Vorgängerarbeit wurde bereits über verschiedene mögliche Simulationsumgebungen geschrieben
%     \item die Bedingungen haben sich leicht geändert
%     \item eine mögliche Software (MuJoCo) ist mittlerweile nicht mehr kostenpflichtig, was damals einer der Gründe war, die gegen diese Software gesprochen haben
%     \item andererseits soll diese Arbeit an die vorangegangene anknüpfen
%     \item wenn die Simulationsumgebung gewechselt würde, würde man im Grunde genommen kaum Ergebnisse der Vorgängerarbeit aufgreifen sondern in vielen Gesichtpunkten von 0 beginnen
%     \item deshalb soll weiterhin Unity verwendet werden
%     \item Unity bietet jedoch die Möglichkeit, die standardmäßige Physics-Engine gegen andere nach dem Plugin-Prinzip auszutauschen
%     \item insofern könnte realistisch und mit geringem Aufwand in Betracht gezogen werden, MuJoCo als Physics-Engine in Unity einzubinden, um somit das Ergebnis des Trainings durch eine andere/verbesserte Physiksimulation zu verbessern
%     \item auch wäre es möglich, die Ergebnisse der verschiedenen Umgebungen zu vergleichen
%     \item da jedoch keine Übertragung auf einen realen Roboter erfolgt, dürfte in diesem Kontext kein spürbarer Unterschied zu beobachten sein / ein beobachteter Unterschied könnte nicht hinsichtlich seiner Aussagekraft eingeordnet werden
% \end{itemize}

\section{Geplante Realisierung}
\label{sec:realisierung}
Die Umsetzung der Arbeit lässt sich in mehrere Aufgabenpakete gliedern.
Diese sollen folgend beschrieben werden.

\subsection{Rekonstruktion}
Der erste Schritt besteht darin, die Simulationsumgebung und Lernergebnisse der vorherigen Arbeit zu rekonstruieren.
Diese Rekonstruktion bringt Probleme mit sich.
Einige der verwendeten Komponenten sind einer starken Entwicklung unterlegen -- insbesondere das erst 2017 vorgestellte ML-Agents Toolkit, welches sich zum Zeitpunkt der Bearbeitung des Basisprojekts noch in der Pre-Release-Phase befand \cite{mlagentsHistory}.
Deshalb sind auf jeden Fall Änderungen nötig, um die bestehenden Ergebnisse überhaupt sichten zu können.
Außerdem wird die Umgebung auf den aktuellen Stand der Technik migriert, um von potenziellen Verbesserungen im verwendeten Tooling profitieren zu können.
Auch wird damit eine zukunftssicherere Basis geboten, auf der die Ergebnisse dieser Arbeit weitergenutzt werden können.

\subsection{Entwurf der Pfadplanung}
Im Anschluss an die Konstruktion einer verwendbaren Simulationsumgebung muss ein Format entworfen werden, wie dem Roboter ein Pfad mitgeteilt werden kann, dem dieser folgen soll.
Ein Pfad besteht aus mathematischer Sicht aus einer geordneten Aufreihung an Punkten.
Der Roboter muss diese der Reihe nach ansteuern.
Wie in \autoref{sec:probleme} bereits ausgeführt, muss verhindert werden, dass der Roboter einen vorgegebenen Pfad auswendig lernt.
Als logische Folgerung muss bei jeder Trainingsepisode ein zufälliger Pfad vorgegeben werden.
Beim Verflogen eines Pfades ist zu jeder Zeit nur der als nächstes anzusteuernde Punkt von Relevanz.
Für das Training ergibt dies Parallelen zu einem Beispielszenario des ML-Agents-Toolkits.
Beim \emph{Crawler-Example} muss eine Kreatur lernen, ein zufällig spawnendes Ziel (sogenannte \emph{DynamicTargets}) zu erreichen.
Kommt der Crawler mit einem DynamicTarget in Berührung, teleportiert es sich an einen zufälligen Ort \cite{crawlerExample}.
Dabei tauch die DynamicTargets als kleine Würfel in der Umgebung des Crawlers auf.
Die DynamicTargets können auch für die Pfadplanung des Roboters verwendet werden.
Die für das Training notwendigen, zufälligen Pfade bilden sie bereits automatisch ab.
Um dem trainierten Modell des Crawlers einen festen Pfad vorgeben zu können, muss lediglich eine kleine Modifikation der DynamicTargets vorgenommen werden, um diese in einer festgelegten Reihenfolge anstatt zufällig erscheinen zu lassen.
Dieses Prinzip soll auf diese Studienarbeit übertragen werden und es so dem Roboter ermöglichen, einem geplanten Pfad zu folgen.

\subsection{Anpassung der Reward-Funktion}
Bislang wird der Roboter nur für eine zurückgelegte Distanz entlang der x-Achse belohnt und erhält eine Bestrafung, wenn er sich auf den Rücken dreht.
Im ersten Schritt soll das Laufverhalten des Roboters stabilisiert werden, damit sich dieser nicht mehr springend fortbewegt.
Dazu kann etwa in der Reward-Funktion ein Faktor eingebracht werden, der den Roboter mit einem, an der Neigung der Zentralplatte skalierten, Wert bestraft.

Um den Roboter im Zuge der Pfadplanung dazu zu bringen, zum nächsten Zielpunkt zu laufen, muss die bisherige Belohnung für die Distanz ersetzt werden.
Denkbar sind hierfür mehrere Ansätze.
Eine Möglichkeit besteht darin, die Distanz zwischen Roboter und Zielpunkt zu bestimmen und mit der Distanz vor der letzten Aktion zu vergleichen.
Die andere Möglichkeit ist, die konkrete Bewegungsrichtung des Roboters zu bestimmen und ein Produkt mit der Geschwindigkeit des Roboters zu bilden.
Dies ermöglicht eine potenziell feinere Gewichtung der Faktoren.
Der erste Berechnungsansatz stellt hingegen eine wesentlich kleinere Veränderung zur Ausgangssituation dar, weshalb hierbei die möglichen Fehlerquellen besser eingegrenzt werden können.
Es sollen beide Ansätze ausprobiert und miteinander verglichen werden.

\subsection{Ergänzen von Hindernissen}
Als letzter Schritt sollen noch Hindernisse ergänzt werden.
Hierfür können einfache Kisten in Unity verwendet werden.
Auch die Position dieser Hindernisse darf natürlich nicht von Trainingsalgorithmus auswendig gelernt werden, weshalb die Kisten in jeder Trainingsepisode zufällig platziert werden müssen.
Wenn diesen Kisten nun Kollisionsmodelle hinzugefügt werden, kann der Roboter automatisch nicht mehr durch diese Hindernisse hindurchgehen.
Die größte Herausforderung wird voraussichtlich auch hier die Adaption der Reward-Funktion sein, damit diese den Roboter nicht daran hindert, den Pfad zu verlassen.
Gleichzeitig soll der Roboter auch nicht dazu animiert werden, den vorgegebenen Pfad großräumig zu verlassen.
Die Anpassungsmöglichkeiten sind hier jedoch relativ eingeschränkt.
Möglich ist, bei jedem Simulationsschritt eine kleine Strafe zu vergeben, mit dem Ziel, den Roboter zu animieren, nicht vor einem Hindernis stehenzubleiben, weil dies auf lange Sicht eine sehr schlechte Belohnung für ihn bedeutet.
Andererseits ist es auch wichtig, eine hohe Entropie für den Roboter zu haben, damit dieser nach Wegen um das Hindernis sucht, anstatt dort im lokalen Maximum steckenzubleiben.

% \begin{enumerate}
%     \item alte Umgebug und Lernergebnisse des Roboters rekonstruieren; bringt Probleme mit sich, da einige der verwendeten Komponenten einer starken Entwicklung unterliegen/unterlagen, weshalb potenziell Anpassungen vorzunehmen sind, um die alte Umgebung weiterhin verwenden zu können oder die Umgebung auf einen aktuellen Stand der Technik migriert werden sollte, um von Verbesserungen im verwendeten Tooling zu profitieren und eine zukunftssichere Basis zu bieten
%     \item Format entwerfen, wie dem Roboter ein Pfad mitgeteilt werden kann, dem dieser folgen soll; ein Pfad wird hierbei voraussichtlich aus mehreren Punkten bestehen, die sich entlang seines Verlaufs befinden
%     \item Belohnungsfunktion anpassen; oben erwähnt Anpassungen hinsichtlich Laufstabilität; außerdem muss ein Abweichen vom direktesten möglichen Weg bestraft werden / zu einer ausbleibenden oder sehr geringen Belohnung führen
    
%     Bestrafung pro Schritt als mögliche Lösung; dies sollte dem Roboter ein inhärentes Interesse verleihen, möglichst schnell wieder auf den richtigen Pfad zurückzukehren und seine Aufgabe zu erledigen
%     \item Hindernisse ergänzen; hierfür sollte die Kollisionserkennung der Unity-Engine verwendet werden können; mithilfe des Hinzufügens von Kollisionsmodellen für in der Simulationsumgebung platzierten Objekte, sollte der Roboter automatisch nicht mehr durch Hindernisse hindurch gehen können; größte Herausforderung sollte hier die Adaption der Belohnungsfunktion werden, damit diese den Roboter nicht daran hindert, den Pfad zu verlassen, das Hindernis zu umgehen und anschließend auf den Pfad zurückzukehren und eine deutlich gesteigerte Belohnung zu erhalten; gleichzeitig darf der Roboter sich nicht zu frei im Raum bewegen, sondern sollte so dicht wie möglich am vorgegebenen Pfad bleiben
% \end{enumerate}
\include{Inhalt/04_Inhalt/0500_implementierung}
\chapter{Bewertung der Ergebnisse}
\begin{itemize}
    \item Unterschiede zu Crawler-Example
    \item denkbar sind Fehler in der Simulation der Servomotoren, die zu einem unrealistischen Ergebnis führen
    \item evaluieren, um welche Sensoren der Roboter in der Realität sinnvoll ergänzt werden könnte und dem Roboter mehr Informationen über sich selbst bereitstellen; Mangel an Informationen wurde hier als Grunddefinition gesetzt, ist jedoch elementarster Unterschied zu vergleichbaren Projekten, welche allerdings deutlich erfolgreichere Ergebnisse vorweisen
    \item herausfinden, warum Policy Loss / Value Loss so hoch sind und wie das Training nachhaltig stabilisiert werden könnte
\end{itemize}
\chapter{Fazit / Future Work}\todo{ein Titel oder ggf zwei Kapitel?}

% ---- Literaturverzeichnis
\cleardoublepage
\renewcommand*{\chapterpagestyle}{plain}
\pagestyle{plain}
\pagenumbering{Roman}                   % Römische Seitenzahlen
\setcounter{page}{\numexpr\value{savepage}+1}
\printbibliography[title=Literaturverzeichnis]

% ---- Anhang
\appendix
\chapter{Hinweis}
Der vollständige Quellcode sämtlicher Trainingsschritte ist frei verfügbar unter \url{https://github.com/MobMonRob/HindernisumfahrungRLStudien} sowie unter \url{https://github.com/yschiebelhut/spider_bot_rl_training}.
Weiterhin enthält \url{https://github.com/yschiebelhut/spiderbot-training-data} die für Linux kompilierten Trainingsumgebungen, die trainierten Modelle und alle notwendigen Daten, um die Trainingsprozesse mit Tensorboard nachvollziehen zu können.

\paragraph{Wichtig:} Zur Durchführung der Inferenz eines bestimmten Modells sollte das Haupt-Repository möglichst auf dem entsprechenden Commit ausgecheckt sein, auf dem das Training stattgefunden hat.
(Die Numerierungen in den Commits und den Run-IDs geben darüber Aufschluss.)
Dies ist wichtig, da insbesondere die Größe des Observation Space als auch die Observations selbst mit den beim Training verwendeten Daten übereinstimmen müssen, weil es ansonsten zu Fehlern bei der Interpretation des neuronalen Netzes kommt.


\chapter{ML-Agents Training Konfiguration}
\label{anhang:trainer-config}
\begin{figure}[H]
    \lstinputlisting[
        label = code:trainer-config-yaml
    ]{Code/new-trainer-config.yaml}
\end{figure}
%\clearpage
%\pagenumbering{Roman}  % römische Seitenzahlen für Anhang

\newpage
\end{document}
